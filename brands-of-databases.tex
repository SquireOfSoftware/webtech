\title{Database management systems}

\chapter{MySQL}
http://www.datasciencecentral.com/profiles/blogs/history-of-mysql
https://en.wikipedia.org/wiki/MySQL
https://www.youtube.com/watch?v=iFUmdHjjon4
- on architecture
https://www.safaribooksonline.com/library/view/high-performance-mysql/9781449332471/ch01.html

\section{What is it?}
- relational database management system \(RDMS\)
- open source
- owend by Oracle at the moment
- written in C and C++
- quite popular though development has "stagnated" since it was taken over by Oracle

\section{What is it used for?}
- developed to have a "faster" interface than its mSQL system
- redeveloped the SQL interface with the same access points as mSQL

\section{How does it work?}
- works as a standard RDMS
- the general solution that it solves is having an easy access to all tables within the system, the access is provided via the SQL syntax
- can scale out to different machines via sharding, different "parts" of a table are given to different instances to hold - enterprise editions support sharding automatically
- isolates its data retrieval/storage processes from its querying processes - as a result you can easily plug in your own storage engine at the back
- each request gets a "thread" and these requests are cached so future requests can reuse past requests
- uses read/write locks to prevent multiple users trying to access the same record
- can apply a variety of different locks depending on the situation on the storage engine - table locks \(lock the whole table\) or row locks \(lock the whole record\)
- live table modifications such as adding a new column will cause the table to be locked for the duration that the new column is being added

\section{What does it work with?}
- wide support for a variety of operating systems
- wide support for a variety of languages \(PHP, Java, C, C++, Perl, Python etc.\)

\section{What unique features does this have?}
- wide support for a variety of operating systems
- can be multi-threaded
- has a community and a proprietary versions
- quite widely used so that is a lot of support

\chapter{PostgreSQL}
https://www.postgresql.org/
https://en.wikipedia.org/wiki/PostgreSQL
https://www.digitalocean.com/community/tutorials/sqlite-vs-mysql-vs-postgresql-a-comparison-of-relational-database-management-systems
- on architecture
https://cisc322.files.wordpress.com/2010/10/conceptual_architecture_of_postgresql.pdf
https://vladmihalcea.com/2017/03/01/how-does-mvcc-multi-version-concurrency-control-work/

http://bajis-postgres.blogspot.com.au/2013/10/postgresql-architecture.html

\section{What is it?}
- object relational database management system \(ORDMS\)
- transactional
- open source
- written in C
- designed for "high volume environments"

\section{What is it used for?}
- according to the documentation - high volume environments
- highly compliant with the SQL standard and enables concurrency without read locks
- used when data integrity is a requirement \(ensuring that data is the same as when it goes in\)
- can be built upon to provide custom functions and features

\section{How does it work?}
- process-based - one database session per CPU
- uses MVCC - every read request is basically a "snapshot" of a table and is not locked when writing to the database
- each entry has an insertion timestamp and a deletion timestamp - this enables the snapshots to take place should things get out of wack
- note that you will need to "vacuum" up the deleted entries when the database is idle to release the space

\section{What does it work with?}
- support Java, C, C++, .Net, Perl, Python, Rub

\section{What unique features does this have?}
- concurrency control - AKA MVCC a core feature of PostgreSQL which provides concurrent read access without the need for "locks" on cells, it works by taking "snapshots" of the database at given time intervals when needed, note that this does introduce extra storage requirements for all of these "snapshots"
- ACID compliant
- asynchronous replication
- point in time recovery
- compliant to the SQL:2008 standard - most compliant
- table inheritance
- huge community support


\chapter{MongoDB}
https://www.mongodb.com/nosql-explained
https://en.wikipedia.org/wiki/MongoDB
https://db-engines.com/en/system/MariaDB%3BMongoDB
http://www.aptuz.com/blog/is-postgres-nosql-database-better-than-mongodb/
- problems with mongodb
http://developer.olery.com/blog/goodbye-mongodb-hello-postgresql/

\section{What is it?}
- type of NoSQL database
- open source document-oriented database - classified as a "document store"
- high availability - can defer writing to disk until later - though can lose information
- developed by MongoDB Inc.

\section{What is it used for?}
- agile projects, relational databases don't scale well with agile development, each new requirement means a redeployment of a database schema to match the new requirements
- naturally shard or spread data around to different servers
- used with environments where concurrency is needed and data integrity and consistency \(issues with evolving data and schemaless structures\) aren't

\section{How does it work?}
- the database system is divided into a querying system, a virtual data model and then a storage engine
- all data are stored as "documents" encoded into BSON \(Binary JSON\) - this is the data model
- data for one application is generally kept together in a single "document"
- the querying system can support JS queries
- sharding occurs automatically - when certain nodes go down, it can automatically determine who will be the next "primary" node

\section{What does it work with?}
- supports: Java, .NET, NodeJS, Ruby, Javascript, Python, PHP etc.

\section{What unique features does this have?}
- does not support SQL
- autosharding
- information for a particular application is collected together as a "document" - makes it a lot easier to access information from a programmatic view

\chapter{Cassandra}
http://cassandra.apache.org/
https://www.youtube.com/watch?v=0UA27d12urM
https://www.youtube.com/watch?v=ZzFCfH8e3QA
http://docs.datastax.com/en/dse/5.1/dse-arch/
https://en.wikipedia.org/wiki/Apache_Cassandra
http://vldb.org/pvldb/vol5/p1724_tilmannrabl_vldb2012.pdf
https://medium.com/@mustwin/cassandra-from-a-relational-world-7bbdb0a9f1d

\section{What is it?}
- distributed database - classified as a "wide column store"
- keeps information spread out across multiple machines
- need to think about data model first before you use it
- built on Java

\section{What is it used for?}
- ensuring data is fault resistant
- clusters can keep information up for long periods of time
- generally used for things like banking
- used for "transaction" - lots of concurrent traffic

\section{How does it work?}
- cassandra instances are usually placed in "clusters" or groups of other instances
- each cassandra instance shares information \(what they have\) to each other, propagate changes to at least another node (dependong on the configuration) - this is called "gossip" - allows for state tracking
- a "snitch" records the ip addresses to the physical locations of the instances
- data replication can depend on the architecture that is being deployed, can have all nodes having everything or you can just have groups of instances having the same amount of data

\section{What does it work with?}
- Java
- Python
- Node
- Go
- C++

\section{What unique features does this have?}
- doesn't support table joins
- can provide fast write operations
- high availability
- uses CQL instead of SQL

\chapter{HBase}
https://en.wikipedia.org/wiki/Apache_HBase
https://hbase.apache.org/
https://hortonworks.com/apache/hbase/
https://mapr.com/blog/in-depth-look-hbase-architecture/
https://www.dezyre.com/article/overview-of-hbase-architecture-and-its-components/295
http://www.infoworld.com/article/2610656/database/big-data-showdown--cassandra-vs--hbase.html

\section{What is it?}
- non-relational distributed database - classified as a "wide-column store" - competes with Cassandra
- open source
- supported by Apache
- built with Java
- developed as part of Apache Hadoop - runs on top of HDFS \(Hadoop Distributed File System\)
- does not natively support SQL but can have extensions which provide SQL functionality
- more of a "data store" than a DBMS - missing some management features

\section{What is it used for?}
- works with Hadoop and provides storage for "sparse data"
- Hadoop doesn't handle large amounts of reads and writes, HBase was developed to assist with this
- created to provide real time \(or close to real time\) access to data in Hadoop - very fast random access
- aims to hold large tables
- strong consistency - all readers will see the same value
- built-in recovery - via WAL
- has some compliancy with SQL

\section{How does it work?}
- composed of three main components: Zookeeper, Region servers and HMaster
- Region servers serve data for reads and writes
- Zookeeper is for tracking the server state of each state of every Region server, including the HMaster servers
- HMaster for coordinating region servers, administration and monitoring - ensuring that loads are distributed evenly across the regons
- Zookeeper tracks the state via heartbeats
- clients/users query the Zookeeper to get the Meta table which tells it about all the locations of the Region servers
- Region servers use HDFS \(Hadoop Distributed File System\)
- Region servers use WAL \(write ahead logs\) for new data, have a read cache called BlockCache, write cache called MemStore and stores data as Hfiles
- WAL can be "replayed" as it contains the running log of database updates

\section{What does it work with?}
- works with: C, C#, C++, Java, PHP, Python, Scala, Groovy

\section{What unique features does this have?}
- auto sharding when things get to big, it automatically spreads out the data - region splitting - dividing a table in half and having two tables that contain two halves of a range
- very fast access

\chapter{Microsoft Access}
https://msdn.microsoft.com/en-us/library/cc749861.aspx
https://en.wikipedia.org/wiki/Microsoft_Access
https://www.linkedin.com/pulse/20141204231202-292713093-microsoft-access-vs-microsoft-sql-server
https://db-engines.com/en/system/Microsoft+Access%3BPostgreSQL

\section{What is it?}
- relational database
- works well with Microsoft products
- can integrate with Microsoft SQL Server
- it is not excel spreadsheet
- can only store up to 2Gb per database
- built with C++
- commercial

\section{What is it used for?}
- generally used for small local database instances
- used for some microsoft products
- relational database maintained by Microsoft
- originally developed to allow users to access data from anywhere, whether its ASCII or a SQL Server

\section{How does it work?}
- has a backend and a front end
- backend is a Jet/Ace database engine - does the storing of information
- front end is a RAD \(Rapid Application Development\) tool - interacts with db to generate reports
- data is locked per row

\section{What does it work with?}
- VBA - Visual Basic for Applications
- ODBC access via Jet/Ace database engine
- ASP.NET
- PHP
- Java

\section{What unique features does this have?}
- only works for Windows
- very easy to set up interactions with Access

\chapter{Redis}
https://redis.io/
https://redislabs.com/why-redis/
https://db-engines.com/en/system/PostgreSQL%3BRedis
http://qnimate.com/overview-of-redis-architecture/
https://redis.io/presentation/Redis_Cluster.pdf
http://oldblog.antirez.com/post/MongoDB-and-Redis.html
https://stackoverflow.com/questions/14989390/why-should-i-use-redis-when-i-have-postgresql-as-my-database-for-django

\section{What is it?}
- open source managed by Redis Labs
- an in-memory database that uses key-value store
- a popular key-value database - categorised under NoSQL
- built with C

\section{What is it used for?}
- generally used for caching, message queues
- used by Twitter
- fast read and write times \(since everything is in memory\) at the cost of more complicated queries
- can be used in conjunction with persistent databases such as PostgreSQL - speeds up access and storing only when you need to
- attempts to approach the data structure problem differently to that of databases like MongoDB

\section{How does it work?}
- all data is stored in memory \(RAM\)
- access can be provided through snapshotting and semi-persistance via writing to disk from time to time
- server side scripting via LUA
- divided into two sections: redis client and redis server
- redis client is how you interact with the redis server
- redis server is where the data is stored
- can set up a cluster of redis servers, the cluster nodes each talk to each other via pings \(hey there\) and pongs \(nice to meet you\)
- nodes in a cluster are very chatter and "gossip" about the state of one another to each other
- can auto-shard if a particular key-value set gets too big
- using a redis-trib to manage the cluster

\section{What does it work with?}
- C, C++, C#, Java, Javascript, etc.

\section{What unique features does this have?}
- can support Abstract Data Types
- schema free
- cannot store large data due to memory limitations
- poor native recovery - if database crashes, since most of it is in memory, most of it will be lost

\chapter{ObjectDB}
http://www.objectdb.com/
http://www.objectdb.com/java/jpa/persistence/managed
https://stackoverflow.com/questions/5291950/is-objectdb-production-ready
http://www.jpab.org/EclipseLink/PostgreSQL/server/ObjectDB/ObjectDB/server.html

\section{What is it?}
- object database for Java
- written in Java
- uses two standard Java APIs to interact with the database - JPA \(Java Persistence API\) or JDO \(Java Data Objects\)
- standards are built into the database, so no external mapping is required since the mapping is made for you

\section{What is it used for?}
- provides a nice interface that hides SQL queries from the user

\section{How does it work?}
- works with Java out of the box, minimal database configurations are required for Java to talk to ObjectDB
- all you need to do is set up the entity objects and library references and you can read and write to the database
- very little information can be found on how it works

\section{What does it work with?}
- Java

\section{What unique features does this have?}
- works well with Java
- minimal integration configurations required

\chapter{SQLite}
https://en.wikipedia.org/wiki/SQLite
https://db-engines.com/en/system/PostgreSQL%3BSQLite
https://www.sqlite.org/arch.html
https://www.digitalocean.com/community/tutorials/sqlite-vs-mysql-vs-postgresql-a-comparison-of-relational-database-management-systems
http://use-the-index-luke.com/blog/2014-05/what-i-learned-about-sqlite-at-a-postgresql-conference

\section{What is it?}
- an embedded relational database management system \(RDMS\)
- public domain and open source
- built with C
- uses files to store data

\section{What is it used for?}
- generally used for embedded applications that require storage like web browsers
- avoid using this if there needs to be multiple users or applications that reqauire high write volumes
- "SQLite is a replacement for fopen"

\section{How does it work?}
- stores information into a file - one file per database
- needs ODBC or JDBC to interact with the data
- concurrent access is achieved by locking files
- query statements are transformed into bytecode and then fed into a VM, which then talks to the appropriate OS environemnt to store data

\section{What does it work with?}
- works with: C, C++, Java, Javascript, Python, Ruby etc.

\section{What unique features does this have?}
- does not require a server to operate
- database is basically a file that can be shared around
- no user management

\chapter{Flat file database}
https://www.techopedia.com/definition/7231/flat-file-database-database
https://en.wikipedia.org/wiki/Flat_file_database
http://www.valentina-db.com/dokuwiki/doku.php?id=valentina:products:adk:v4rev:howto:databases_from_zero:what_is_in_a_database_and_why_excel_isnt_a_database

\section{What is it?}
- physical data model
- all data is stored in a single "table"
- each row is an entry or record in the database
- data is generally stored in plain text or in binary

\section{What is it used for?}
- generally for configuration files

\section{How does it work?}
- file must be completely read before data can used
- each entry/record must have its relevant column information separated by a delimiter, which might be a comma or special characters that allow for "cells" to be identified

\section{What does it work with?}
- anything that can interpret plain text

\section{What unique features does this have?}
- flexible - can work with anything that can interpret plain text
- can be very large

\chapter{Overall comparison of database management systems}

- MySQL
- PostgreSQL
- MongoDB
- Cassandra
- HBase
- Microsoft Access
- Redis
- SQLite


\title{Storage engines}
https://mariadb.com/kb/en/mariadb/choosing-the-right-storage-engine/

\chapter{InnoDB}
https://dev.mysql.com/doc/refman/5.7/en/innodb-transaction-model.html
https://en.wikipedia.org/wiki/InnoDB
https://www.percona.com/live/mysql-conference-2013/sessions/innodb-architecture-and-performance-optimization

- designed to combine "multi-versioning" and 2PL \(two phase locking\)
- transaction database
- used in MySQL
- all data is stored in "tablespace" and tablespace change must be accepted before recordi s overwritten
- tablespace is never shrunk
- can't downgrade once optional features such as separating directory, logs and tablespaces
- has main thread to handle commands
- has io threads to do reads, writes, inserting into the buffer, adding to the logs
- has buffer pool which contains, database cache, insertion buffers, locks

\chapter{WiredTiger}
http://source.wiredtiger.com/2.6.0/architecture.html

- used in mongoDB

\chapter{InMemory}


\chapter{Microsoft Jet Database Engine}
https://en.wikipedia.org/wiki/Microsoft_Jet_Database_Engine

\chapter{MyISAM}


\chapter{Other information}

\section{Database vs Storage engine}
- Storage engines, these are the actual data is stored in, the databases that were discussed above are a database management system
